\begin{table}[!h]
\centering
\captionof{table}{Decomposition of differences between the French and U.S.\ economies}
\label{tab:France_vs_Spain} 
\begin{tabular}{l c c c c c}
\hline \hline
\addlinespace
 \hspace{100pt} &    Total  & $A$ & $\delta$ & $b$ & $F$   \\
 \addlinespace
 Reference param.\ value (U.S.) & &0.60&0.0024&0.38&0\\
 Comparison param.\ (France) & &0.30&0.0008&0.50&1.37\\
 Param.\ differential &  &-0.30&-0.0016&0.12&1.37\\
 \addlinespace
 \multicolumn{6}{l}{\textit{Labor-market stocks differential (percentage-point)}}  \\
 \hspace{5pt} Unem.\ rate &3.57&0.70&4.93&0.63&3.62\\
 \addlinespace
 \textit{Transition rates (p.p.)}  &   &   &   &   &  \\
 \hspace{5pt} UE           &-21.77&-4.55&-21.25&-12.51&-21.63\\
 \hspace{5pt} EU           &-0.81&-0.33&-0.29&-1.08&-0.78\\
\addlinespace
\hline \hline
\end{tabular}
\caption*{ \footnotesize Notes: Decomposition of differences in aggregate equilibrium outcomes between the ``French'' and U.S.\ baseline model economy. The first column reports the total difference in equilibrium outcomes between these two economies. The subsequent columns report the difference in outcomes after counterfactually imposing a given parameter in the U.S.\ baseline to be equal to its counterpart in the French economy, i.e., a measure of the total difference in equilibrium outcomes attributable to this parameter. The parameters under focus are the matching efficiency $A$ , the exogenous probability of separation $\delta$, the non-work income level $b$, and firing costs $F$. The first row indicates parameter values in the baseline (U.S.) reference economy. The second row indicates the difference in parameter values between the baseline and French model economies. }
\end{table} 