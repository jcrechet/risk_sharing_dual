\begin{table}[!h]
\centering
\captionof{table}{Decomposition of differences between the French and Spanish economies}
\label{tab:France_vs_Spain} 
\begin{tabular}{l c c c c c}
\hline \hline
\addlinespace
 \hspace{100pt} &    Total  & $\delta$ & $b$ & $F$ & $\phi$   \\
 \addlinespace
 Reference param.\ value (French) & &0.0008&0.50&1.37&0.045\\
 Comparison param.\ (Spanish) & &0.0019&0.53&1.65&0.030\\
 Param.\ differential &  &0.0011&0.03&0.28&-0.015\\
 \addlinespace
 \multicolumn{6}{l}{\textit{Labor-market stocks differential (percentage-point)}}  \\
 \hspace{5pt} Unem.\ rate &4.32&1.86&2.96&4.30&3.92\\
 \hspace{5pt} TC emp.\ share           &7.52&4.48&6.62&7.47&4.46\\
 \addlinespace
 \textit{Transition rates (p.p.)}  &   &   &   &   &  \\
 \hspace{5pt} UE           &-0.97&-0.92&-0.06&-0.94&-0.98\\
 \hspace{5pt} EU           &0.50&0.14&0.41&0.49&0.44\\
\addlinespace
\hline \hline
\end{tabular}
\caption*{ \footnotesize Notes: Decomposition of differences in aggregate equilibrium outcomes between the ``French'' and ``Spanish'' model economy. The first column reports the total difference in equilibrium outcomes between these two economies. The subsequent columns report the difference in outcomes after counterfactually imposing a given parameter in the French model economy to be equal to its counterpart in the Spanish economy, i.e., a measure of the total difference in equilibrium outcomes attributable to this parameter. The parameters under focus are the exogenous probability of separation $\delta$, the non-work income level $b$, firing costs $F$, and the regulatory restriction on the duration of temporary contracts, $\phi$. The first row indicates parameter values in the French reference economy. The second row indicates the difference in parameter values between the Spanish and French model economies. }
\end{table} 