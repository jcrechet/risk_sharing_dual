\begin{table}[!h]
	\centering
	\captionof{table}{Benchmark parameter values}
	\label{tab:param:benchmark}
	\begin{tabular}{l l c}
		\hline \hline
		\addlinespace
		\textit{Preset} & & \\ 
		$\beta$ & Discount factor & 0.996 \\ 
		$\sigma$ & Relative risk aversion & 2.5 \\ 
		$\eta$ & Elasticity of matching & 0.5 \\ 
		$\gamma$ & Worker's bargaining power & 0.5 \\ 
		$\delta$ & Exogenous separation probability & 0.0024 \\ 
		$F$ & Firing costs & 0 \\ 
		\addlinespace
		\textit{Internal} & & \\ 
		$b$ & Non-work income & 0.383 \\ 
		$A$ & Matching efficiency & 0.600 \\ 
		$\lambda$ & Probability of idiosyncratic shock & 0.068 \\ 
		$\sigma_x^2$ & Log match-quality variance & 0.204 \\ 
		$\mu_x$ & Log match-quality mean & -0.102 \\ 
		$\kappa$ & Vacancy posting cost & 2.096 \\ 
		\addlinespace
		\hline \hline
	\end{tabular}
	\caption*{\footnotesize Notes: Parameter values in the benchmark U.S.\ calibrated economy. The exogenous probability of separation $\delta$ is set following \cite{jung_kuhn:JEEA:2019}. The other preset parameters are standard. The targets for the internally calibrated parameters are an unemployment rate equal to 5.6\%, an aggregate EU probability of 2\%, non-work income $b=0.4$ (\cite{shimer:2005:AER}), and relative separation rates by tenure estimated from the Job Tenure Supplement of the CPS (less than one year of tenure to 1-15 years). Moreover, $\mu_x = -\sigma_x^2/2$. The value of $\kappa$ is set to be consistent with the normalization that labor-market tightness $\theta=1$ (\cite{shimer:2005:AER}). In addition, the model is calibrated in partial equilibrium with the value of unemployment $U$ exogenous and the parameter $b$ is computed accordingly (see the text for details). The model fit to the calibration targets is shown in Table \ref{tab:targets}, Panel a).}\end{table}
