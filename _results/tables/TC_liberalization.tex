\begin{table}[!h]
\centering
\captionof{table}{The employment effect of temporary contracts}
\begin{tabular}{l c c c c}
\hline \hline
\addlinespace
 & \multicolumn{2}{c}{ \textit{French institutions} } & \multicolumn{2}{c}{ \textit{Spanish institutions} } \\
 & \textit{No restrictions}  & \textit{Pre-reform} &  \textit{No restrictions}  & \textit{Pre-reform} \\
 & $\phi_0=0$ & $\phi_0=0.55$ & $\phi_0=0$ & $\phi_0=0.72$ \\
 \addlinespace 
 Unemp.\ rate (\%)                                &9.21 & 9.22 & 13.54 & 13.16\\
 TC emp.\ share                              &12.66 & 5.56 & 20.17 & 5.65\\
 \addlinespace
UE rate (\%)           &11.75 & 10.76 & 10.79 & 9.36\\
UP rate                  &1.71 & 6.27 & 1.48 & 6.79\\
 \addlinespace
EU rate (\%)           &1.19 & 1.09 & 1.69 & 1.42\\
PU rate                  &0.62 & 0.85 & 0.83 & 1.20\\
\addlinespace
\hline \hline
\end{tabular}
\label{tab:effect_EPL}
\caption*{ \footnotesize Notes: evaluation of the effect of temporary contracts on employment stocks and flows in the model economies with French and Spanish-type institutions. The columns ``No restrictions'' present equilibrium outcomes (in percentage) in reference economies without hiring restrictions on temporary contracts, i.e., $\phi_0=0$; the columns ``Pre-reforms'' present outcomes in counterfactual economies with $\phi_0$ that is set to generate an employment share of temporary jobs equal to 5.5\%, equal to the average in Continental Europe in 1983 (\cite{faccini:2014:EJ}).}\end{table}